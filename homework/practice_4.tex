%% Preámbulo
%% ----------------------------------------------------------------------------
\documentclass[12pt,a4paper]{practice}
\usepackage[spanish]{babel}
\usepackage[utf8]{inputenc}
\usepackage[T1]{fontenc}
\usepackage{enumitem}
\usepackage{hyperref}
\usepackage{luximono}
\usepackage{textcomp}
\usepackage{graphicx}
\usepackage{amstext}
\usepackage{caption}
\usepackage{charter}

%% Settings
%% ----------------------------------------------------------------------------
% hyperref
\hypersetup{
    pdftitle={Atm\'{o}sferas Estelares - Pr\'{a}ctica 1},
    pdfauthor={Mart\'{i}n Josemar\'{i}a Vuelta Rojas},
    pdfpagelayout=OneColumn,
    pdfnewwindow=true,
    pdfdisplaydoctitle=true,
    pdfstartview=XYZ,
    plainpages=false,
    unicode=true,
    bookmarksnumbered=true,
    bookmarksopen=true,
    bookmarksopenlevel=3,
    breaklinks=true,
    colorlinks=true,
    pdfborder={0 0 0}
}

% graphicx
\graphicspath{{resources/img/}}

% caption
\captionsetup{
    labelfont=bf,
    textfont=it,
    justification=centering,
    width=0.9\textwidth,
}

%% Definicion de comandos
%% ----------------------------------------------------------------------------
\makeatletter

%% Fuente de ancho fijo
\renewcommand{\ttdefault}{lmtt}
\renewcommand{\spanishtablename}{Tabla}

\makeatother

%% Documento
%% ----------------------------------------------------------------------------
\begin{document}
    %% Titulo, autor y resumen --------------------------------------------------
    \logo{unmsm.png}
    \university{
        Universidad Nacional Mayor de San Marcos\\
        {
            \scriptsize{
                \textit{
                    \textup{
                        Universidad del Perú, Decana de América}
                    }
            }
        }
    }
    \course{Tópicos Avanzados I - Relatividad General}
    \title{Práctica N\textdegree\ 1}
    \maketitle

    \begin{problem}\label{prob:1}
        Usando la relaciòn entre las bases: $\epsilon^{1}_{a}=\frac{ax^{b}}{ax^{1a}}e_{b}$
            \begin{enumerate}[label=(\alph*)]
                \item Hallar la relación entre las bases de coordenadas cartesianas y las bases de coordenadas cilindricas.
                \item Hallar la relacióm entre las bases de coordenadas cartesianas y las bases de coordenadas esféricas 
                \item Hallar la relación entre las bases de coordenadas cilíndricas y las bases de coordenadas esféricas
             \end{enumerate}  
    \end{problem}

    \begin{problem}\label{prob:2}
        Dado el tensor de levi-civita de cuarto orden, hallar.
            \begin{enumerate}[label=(\alph*)]
                \item $\epsilon^{\alpha\beta\gamma\sigma} \epsilon_{\mu\nu\rho\sigma}$
                \item $\epsilon^{\alpha\beta\gamma\sigma} \epsilon_{\mu\nu\rho\delta}$
                \item $\epsilon^{\alpha\beta\gamma\sigma} \epsilon_{\mu\nu\gamma\delta}$
                \item $\epsilon^{\alpha\beta\gamma\sigma} \epsilon_{\mu\beta\gamma\delta}$
                \item $\epsilon^{\alpha\beta\gamma\sigma} \epsilon_{\mu\alpha\gamma\delta}$
             \end{enumerate}  
    \end{problem}
    
    \begin{problem}\label{prob:3}
        para tensores antisimétricos de $3^{er}$ y $4^{to}$ rango mostrar que:
        \begin{enumerate}[label=\alph*)]
                \item $T_{(\alpha\beta\gamma)}=\frac{1}{3!}(T_{\alpha(\beta\gamma\delta)}+T_{\beta(\alpha\gamma\delta)} + T_{\gamma(\alpha\beta)} )$
                \item $T_{\alpha\beta\delta}= \frac{1}{4!}(T_{\alpha(\beta\gamma\delta)}-T_{\beta(\alpha\gamma\delta)}+T_{\gamma(\alpha\beta\delta)}-T_{\delta(\alpha\beta\gamma)})$
        \end{enumerate}    
    \end{problem}
    
    \begin{problem}\label{prob:4}
        Reducir las cuatro ecuaciones de Maxwell a dos ecuaciones tensoriales.    
    \end{problem}

    \begin{problem}\label{prob:5}
        Muestre que $B^{k}= -\frac{1}{2}e^{\kappa\iota\eta}F_{\iota\eta}$
    \end{problem}
    
    \begin{problem}\label{prob:6}
        Obtener la fuerza de Lorentz en forma tensorial.    
    \end{problem}
    
    \begin{problem}\label{prob:7}
        Usando $F^{1}_{\mu\nu}=\frac{\alpha x^{\alpha}}{\alpha x^{1\mu}}\frac{\alpha x^{\beta}}{\alpha x^{1\nu}}F_{\alpha\beta}$ con $ T^{1} = \gamma(t- \beta x), x^{1} = \gamma(x-\beta t),y^{1}= y, z^{1}= z$, hallar todas las componentes diferentes de cero del tensor $F^{1}_{\mu\nu}$
    \end{problem}
    
    

\end{document}
