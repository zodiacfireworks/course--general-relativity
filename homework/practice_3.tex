%% Preámbulo
%% ----------------------------------------------------------------------------
\documentclass[12pt,a4paper]{practice}
\usepackage[spanish]{babel}
\usepackage[utf8]{inputenc}
\usepackage[T1]{fontenc}
\usepackage{enumitem}
\usepackage{hyperref}
\usepackage{luximono}
\usepackage{textcomp}
\usepackage{graphicx}
\usepackage{amstext}
\usepackage{caption}
\usepackage{charter}

%% Settings
%% ----------------------------------------------------------------------------
% hyperref
\hypersetup{
    pdftitle={Atm\'{o}sferas Estelares - Pr\'{a}ctica 1},
    pdfauthor={Mart\'{i}n Josemar\'{i}a Vuelta Rojas},
    pdfpagelayout=OneColumn,
    pdfnewwindow=true,
    pdfdisplaydoctitle=true,
    pdfstartview=XYZ,
    plainpages=false,
    unicode=true,
    bookmarksnumbered=true,
    bookmarksopen=true,
    bookmarksopenlevel=3,
    breaklinks=true,
    colorlinks=true,
    pdfborder={0 0 0}
}

% graphicx
\graphicspath{{resources/img/}}

% caption
\captionsetup{
    labelfont=bf,
    textfont=it,
    justification=centering,
    width=0.9\textwidth,
}

%% Definicion de comandos
%% ----------------------------------------------------------------------------
\makeatletter

%% Fuente de ancho fijo
\renewcommand{\ttdefault}{lmtt}
\renewcommand{\spanishtablename}{Tabla}

\makeatother

%% Documento
%% ----------------------------------------------------------------------------
\begin{document}
    %% Titulo, autor y resumen --------------------------------------------------
    \logo{unmsm.png}
    \university{
        Universidad Nacional Mayor de San Marcos\\
        {
            \scriptsize{
                \textit{
                    \textup{
                        Universidad del Perú, Decana de América}
                    }
            }
        }
    }
    \course{Tópicos Avanzados I - Relatividad General}
    \title{Práctica N\textdegree\ 1}
    \maketitle

    \begin{problem}\label{prob:1}
        Muestre que el círculo es cubierto por 2 vecindades abiertas.
    \end{problem}

    \begin{problem}\label{prob:2}
        Muestre que la permeabilidad marnética es un tensor y explique su origen fisico    
    \end{problem}
    
    \begin{problem}\label{prob:3}
        Obtenga la transformación de una coordinada para otra del tensor de rango(m,n)    
    \end{problem}
    
    \begin{problem}\label{prob:4}
        Obtenga la transformación de una coordinada para otra del tensor de rango (3,3)
    \end{problem}

    \begin{problem}\label{prob:5}
        ¿Cuántos componentes independientes tiene el tensor simétrico de segundo rango?    
    \end{problem}

    \begin{problem}\label{prob:6}
        ¿Cuántos componentes independientes tiene el tensor anti simétrico de segundo rango?    
    \end{problem}
    
    \begin{problem}\label{prob:7}
        Bajar los indices de un tensor contravariante de cuarto rango    
    \end{problem}
    
    \begin{problem}\label{prob:8}
        Subir los índices de un tensor convariante de quinto rango
    \end{problem}

    \begin{problem}\label{prob:9}
        Muestre que el producto de 2 tensores de Livi-Civita con las indices contraidos de tercer rango es 6
    \end{problem}
    
\end{document}
