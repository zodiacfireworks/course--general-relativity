%% Preámbulo
%% ----------------------------------------------------------------------------
\documentclass[12pt,a4paper]{practice}
\usepackage[spanish]{babel}
\usepackage[utf8]{inputenc}
\usepackage[T1]{fontenc}
\usepackage{enumitem}
\usepackage{hyperref}
\usepackage{luximono}
\usepackage{textcomp}
\usepackage{graphicx}
\usepackage{amstext}
\usepackage{caption}
\usepackage{charter}

%% Settings
%% ----------------------------------------------------------------------------
% hyperref
\hypersetup{
    pdftitle={Atm\'{o}sferas Estelares - Pr\'{a}ctica 1},
    pdfauthor={Mart\'{i}n Josemar\'{i}a Vuelta Rojas},
    pdfpagelayout=OneColumn,
    pdfnewwindow=true,
    pdfdisplaydoctitle=true,
    pdfstartview=XYZ,
    plainpages=false,
    unicode=true,
    bookmarksnumbered=true,
    bookmarksopen=true,
    bookmarksopenlevel=3,
    breaklinks=true,
    colorlinks=true,
    pdfborder={0 0 0}
}

% graphicx
\graphicspath{{resources/img/}}

% caption
\captionsetup{
    labelfont=bf,
    textfont=it,
    justification=centering,
    width=0.9\textwidth,
}

%% Definicion de comandos
%% ----------------------------------------------------------------------------
\makeatletter

%% Fuente de ancho fijo
\renewcommand{\ttdefault}{lmtt}
\renewcommand{\spanishtablename}{Tabla}

\makeatother

%% Documento
%% ----------------------------------------------------------------------------
\begin{document}
    %% Titulo, autor y resumen --------------------------------------------------
    \logo{unmsm.png}
    \university{
        Universidad Nacional Mayor de San Marcos\\
        {
            \scriptsize{
                \textit{
                    \textup{
                        Universidad del Perú, Decana de América}
                    }
            }
        }
    }
    \course{Tópicos Avanzados I - Relatividad General}
    \title{Práctica N\textdegree\ 1}
    \maketitle

    \begin{problem}\label{prob:1}
        ¿Qué es un sistema de referencia?
    \end{problem}

    \begin{problem}\label{prob:2}
        Formule el principio de relatividad de Glileo-Newton
    \end{problem}

    \begin{problem}\label{prob:3}
        Formule el principio de relatividad de Einstein
    \end{problem}
    
    \begin{problem}\label{prob:4}
        Formule los postulados de la relatividad especial    
    \end{problem}
    
    \begin{problem}\label{prob:5}
        En el diagrama del espacio-tiempo grafique la línea del mundo de:
        \begin{enumerate}[label=(\alph*)]
                \item Una partícula en reposo
                \item Una particula sin masa
                \item Una partícula masiva.
                \item Una partícula hipotética taxion.
        \end{enumerate}    
    \end{problem}

    \begin{problem}\label{prob:6}
        Explique el principio de la casualidad en el diagrama del espacio-tiempo    
    \end{problem}

    \begin{problem}\label{prob:7}
        Grafique el cono de la luz para c que tiende al infinito    
    \end{problem}

    \begin{problem}\label{prob:8}
        ¿Qué es el boost y en qué se diferencia de la rotación?
    \end{problem}

    \begin{problem}\label{prob:9}
        En espacio-tiempo de Minkowsky grafique las trayectorias de
            \begin{enumerate}[label=\alph*)]
                \item Una partícula que se mueve con velocidad constante
                \item Una particula que se mueve con aceleración constante
             \end{enumerate}  
    \end{problem}
    
    \begin{problem}\label{prob:10}
        ¿Por qué aparece el horizonte de eventos?    
    \end{problem}
    
    \begin{problem}\label{prob:11}
        ¿Cuál es la diferencia entre la geometria y la topología? ilustre con 2 ejemplos.
    \end{problem}
    
    \begin{problem}\label{prob:12}
        Defina la variedad y presente su grafica    
    \end{problem}
    
    \begin{problem}\label{prob:13}
        Defina el espacio tangente y presente su grafica
    \end{problem}
    
    \begin{problem}\label{prob:14}
        Explique la diferencia entre el tensor y la matriz
    \end{problem}
\end{document}
